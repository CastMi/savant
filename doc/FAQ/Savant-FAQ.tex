\documentclass[11pt]{article}
\setlength{\unitlength}{1in}
\usepackage{url}

\begin{document}


\begin{center}
SAVANT Frequently Asked Questions

Last updated: \today
\end{center}

\tableofcontents

\section{General Info}
\subsection{What is SAVANT?}
SAVANT is a VHDL analyzer and code generator orignally developed at the
University of Cincinnati, and continued by Clifton Labs, Inc.  (See
\url{http://www.ececs.uc.edu/~paw/savant} and
\url{http://www.cliftonlabs.com/savantp.htm} for the respective websites.)
It is designed to be easily extensible by the end user by allowing the
insertion of new back ends into the intermediate form. The intermediate
form that SAVANT supports is a modified version of the open specification
called AIRE IIR.  See
\url{http://www.cliftonlabs.com/savant/aire-spec/iirUS.pdf} for
the original spec from which Savant's intermediate is based.  Unfortunately
the public AIRE effort was discontinued sometime around 2000.  Since then
we have made assumptions about how to fix ambiguities and had to extend the
specification in several ways.  The documentation for the intermediate as
we have implemented it can be found at
\url{http://www.cliftonlabs.com/savant/docs/savant/html/hierarchy.html}.

\subsection{Is SAVANT still alive?}
As of 3/2005, yes.  Work has continued on SAVANT in bursts since it started
in 1995.  One good place to gauge activity is the mailing list archives.
CVS commits are public and can be found at
\url{http://www.cliftonlabs.com/savant/list-archives/savant-devel/}.  For a
full list of the mailing lists available about Savant and related tools see
\url{http://cliftonlabs.com/cgi-bin/mailman/listinfo}.

In 2004 a new plugin mechanism was developed for Savant that allows custom
backends to be loaded as shared libraries by \texttt{scram} - Savant's
front end.  This has profound implications.  For example, it is now
possible to write proprietary backends for Savant and distribute them under
non-LGPL licensing, making Savant an appropriate choice for a commercial
front-end.  Secondly, the C++ code generation code has been moved out of
the front-end completely making Savant's front-end code much cleaner and
easier to understand.

A tutorial (with a working example) for inserting a new backend into the
analyzer is currently available from Clifton Labs' anonymous CVS server
with the following command:
\texttt{cvs -d :pserver:anonymous@cvs.cliftonlabs.com:/usr/local/cvsroot
checkout savant_plugin_tutorial/}

To see just the documentation go to the following link -
\url{http://www.cliftonlabs.com/~dmartin/ExtensionDocumentation.pdf}.

\subsection{TyVis}
One of the backends that is freely available with the SAVANT distribution
is a C++ code generator. It generates code for the University of
Cincinnati's TyVis VHDL support library. TyVis is a C++ library that
supports many of the concepts of VHDL like signals, signal assignment,
etc. and works with a parallel discrete event simulation library, {\sc
WARPED}.  (See
\url{http://www.ececs.uc.edu/~paw/tyvis} and
\url{http://www.ececs.uc.edu/~paw/warped} for details on those projects.)

Using SAVANT, TyVis, and {\sc WARPED}, parallel VHDL simulation across SMP
workstations, distributed machines, or large parallel systems is possible.

\subsection{What could I use SAVANT for?}
Here are some ideas:
\begin{itemize}
\item{Front end tools utilizing VHDL}  If you wanted to develop a
project that uses VHDL as an input language, you could use the SAVANT
analyzer, scram, to process the VHDL.  Scram takes VHDL input and builds an
in-memory tree structure representing the VHDL.  There are several ways in
which you can then develop tools to access this in-memory representation of
the VHDL.  One example that falls in this category is a VHDL to ``some
other language'' translator.  You would need to define how to turn an AIRE
node into the other language, but the front end is all done for you.

Another possibilty is to use the SAVANT front end to allow you to implement
a design rule checker for custom in-house design rules.

\item{Back end tools for emitting VHDL} - Our implementation of the AIRE
specification has the functionality to write out VHDL from the internal
tree form. So if you were to develop a front end for some language and
used our implementation of AIRE as your intermediate, you'd get VHDL
generation for free.  This would allow you to translate ``some other
language'' into VHDL, as an example.

\item{Custom transformations to VHDL itself} - Another thing you can do with
SAVANT is to use it to transform VHDL to VHDL.  Why would you want to do
this?  Several groups using SAVANT have implemented fault simulators by
inserting fault injection nodes at port boundaries.  Other transformations
could be made, to speed simulation, to allow synthesis of non-synthesizable
constructs... 

\item{Simulation of VHDL} - SAVANT, in combination with {\sc WARPED} and
TyVis can be used to simulate VHDL, both sequentially, and in parallel.
More about that can be found in this document at question
\ref{status_section}.

\end{itemize}

\subsection{Where do I get it?}
SAVANT is available for download at:
\url{http://www.cliftonlabs.com/savant/download/2.0-prerelease/}. Clifton
Labs also maintains a mirror of UC's SAVANT ftp site (with old releases
only) at \url{http://www.cliftonlabs.com/savant/download/uc-mirror/}.  As
of this writing the last ``official'' 2.0 prelease was in December 2003,
but there has been a {\bf lot} of activity since then.  Essentially TyVis
has been rewritten, the plugin architecture developed, and much more.
Because of this the simulation backend is in a state of flux.  If you want
to simulate VHDL get the 12/2003 prerelease.  If you are interested in the
front-end and the plugin architecture, follow the directions at
\url{http://www.phoci.com/~dmartin/savant-devel-primer.html} for getting
SAVANT from CVS.

\subsection{Where do I get std\_logic\_1164 and other libraries for
SAVANT?}
\subsubsection{Answer for version 1.0x}
Check out the comp.lang.vhdl FAQ. It's home is
\url{http://www.vhdl.org/comp.lang.vhdl/} See Part I, section 4.7.
\subsubsection{Answer for version 2.0 (prerelease, currently)}
The libraries are distributed with the analyzer, and if the installation is
correct it should be able to find the automatically.

\subsection{What is the status of SAVANT?}\label{status_section}
SAVANT is under active development at Clifton Labs, Inc.
\url{http://www.cliftonlabs.com}  Clifton Labs provides commercial support
to corporate customers, including several major telecom companies, some
government laboratories, and at least one EDA company.  These customers are
finding SAVANT to be an effective solution for them for doing real work.

SAVANT can analyze real designs. It can analyze over 97\% of the Billowitch
test suite, and lots of real code we've thrown at it, including real
designs for whole systems. It's pretty good at analyzing good VHDL. It will
do weird things with bad VHDL sometimes, including allowing illegal
constructs from time to time. We're still working on that, in addition to
other improvements in the whole system.  Since Savant is LGPL feel free to
fix deficiencies that you find and submit patches back to us.

The code generator doesn't cover the language as well as the analyzer. At
this point there is no documentation to say what's covered and what's not.
We've recently started compiling a list; it will be in some future release.
As you find things that don't work, please file a bug against the system at
\url{http://www.cliftonlabs.com/bugzilla}.  Early in 2004 we started a
rewrite of the TyVis backend with the aim of cleaning up the generated code
to the point that an average C++ developer who was familiar with VHDL could
have a chance of understanding the code.

\subsection{Can I pay someone for support if I'm interested in using SAVANT for an
important project?}\label{commercial_support}

Yes.  Commerical support for SAVANT is available from Clifton Labs,
Inc. (\url{http://www.cliftonlabs.com}) Clifton Labs employs several of the
primary authors of SAVANT and is happy to fix bugs, provide training, phone
support, implement new back-ends - you name it.  Please send email to {\tt
savant@cliftonlabs.com} if you have any questions at all about support of
any kind for SAVANT.

\subsection{I've found a bug, what do I do now}

Clifton Labs maintains a bug reporting system at
\url{http://www.cliftonlabs.com/bugzilla}.  To maximize the chances that
your bug gets fixed, you can do the one or more of the following things:
\begin{itemize}
\item File the bug in the bug reporting system.
\item Be detailed in your report about the version of SAVANT you're using,
the operating system you're running on, the compiler you're using, etc.
\item Provide a concise test case that shows the problem.
\item Provide a patch that fixes the problem.
\end{itemize}

Doing these things will move you to the top of the list of bugs that get
fixed.  The best way to ensure that your bugs get fixed is to consider
support from Clifton Labs.  See \ref{commercial_support}.

\section{Building scram}

\subsection{What is scram?}

Scram is the name of the analyzer/code generator in SAVANT. No one can
remember why it's named that way. We like it though.

\subsection{Where do I get PCCTS ?}

You can obtain PCCTS from \url{http://www.polhode.com/pccts.html}.  It's
also part of some linux distributions.  If you're using Linux, follow the
directions for your distribution for finding and installing packages.

\subsection{Where do I get Flex?}

You can ftp Flex from \url{ftp://ftp.gnu.org}. Go to \texttt{non-gnu/flex}.
See too the note above about Linux distributions.

\subsection{Where do I get GNU make?}

You can ftp GNU make from \url{ftp://ftp.gnu.org}. Go to pub/gnu.

\subsection{Compiling scram is slow - what can I do to get it to compile faster?!}
Buy a faster machine! Well, OK. For us, it compiles in about 13 minutes.
That's on a fast Athlon 64 system. A lot of RAM really helps. More than
512M is probably enough for maximum comfort if you're really developing
with SAVANT.

Other tips... Specify the files that you've touched explicitly to make. If
you modified IIRScram\_Declaration.cc then in the savant/src directory, say
make aire/iir/IIRScram/IIRScram\_Declaration.o. Once you've beaten the bugs
out of your new code, then do a make and let the whole thing build.

\section{VHDL Analysis with "scram"}

\subsection{ Why can't SAVANT parse std\_logic\_signed, std\_logic\_unsigned, 
or some other std\_ library?}

It probably can - we've tested with quite a few of them. However, it's not
so simple as we'd like. This is because every vendor releases non-standard,
non-compatible versions of their own libraries. So you need to find versions
of the libraries that "line up". For instance, use all of the libraries from
the same vendor and you should be OK.

The best bet for portability is to work with the actual IEEE libraries. See
Section 1.2 for details.

\subsection{Can scram parse/simulate my design?}

You'll have to try, we don't know.

\subsection{Can scram parse/simulate any design?}

It can parse and simulate a lot of real designs. We have many working
examples. If you're having problems:
\begin{enumerate}
\item See if you can isolate your problem a bit - we get a lot of email.
\item Send us a simplified test case exhibiting the problem and the full error
messages you're seeing, or a description of the problem. You can send the
email to savant-users@ececs.uc.edu if you like.
\end{enumerate}

\section{Simulation}

\subsection{Where is an example of how to build a simulation using SAVANT?}

One comes with the distribution, and it's also online at
\url{www.ececs.uc.edu/~paw/savant/doc/simulationManual/}

\subsection{When I tried to build my simulation, it didn't build. What should I
do?}

Follow the steps in Section 3.2.

\end{document}
